%----------------------------------------------------------------------

% Template for TECHNICAL REPORTS at Inst. for Computer Graphics

% and Vision (ICG), Graz University of Technology, Austria:

% style file 'techrep_icg.sty'

%

% author:  Pierre Elbischger

% email:   pierre.elibschger@icg.tu-graz.ac.at

% created: 13.11.2003

% last revision: 25.11.2003

%----------------------------------------------------------------------

% The template contains a number of LaTeX commands of the form :

%

% \command{xyz}

%

% In order to complete this abstract, fill in the blank fields between

% the curly braces or replace already filled in fields with the

% requested information.

%

% e.g. in order to add an abstract title, replace

% \title{}    with   \title{Evidence of Solitons in Tedium Diboride}

%

% The \author and \address commands can take an optional label

% in square brackets of the form :

%

% \command[label]{}

%

% The text of the abstract should be inserted between the two commands

% \begin{abstract} and \end{abstract}.

%

% Please leave all commands in place even if you don't fill them in.

%

%----------------------------------------------------------------------

% Do not alter the following two lines

\documentclass[12pt,a4paper]{article}               % I'm using a double-sided book style

      

\usepackage{techrep_icg}
\usepackage{cite}

% package 'graphicx' is automatically included depending on the

%   used compiler (latex, pdflatex), don't include it!!!

\def\BibTeX{{\rm B\kern-.05em{\sc i\kern-.025em b}\kern-.08em
    T\kern-.1667em\lower.7ex\hbox{E}\kern-.125emX}}

\begin{document}

%----------------------------------------------------------------------



\reportnr{001}               % Number of the technical report

\title{Discriminative Classifier Parts} % Title of technical report

\subtitle{Unsupervised discovery of mid-level discriminative patches - An evaluation} % Subtitle of technical report (use small letters only)

\repcity{Graz}            % City where the report was created

\repdate{\today}          % Date of creation

\keywords{Report, AKCV KU 2012, visual representation, unsupervised discovery, discriminative patches} % keywords that appear below the abstract



%----------------------------------------------------------------------

% List of authors

%

% List each author using a separate \author{} command

% If there is more than one author address, add a label to each author

% of the form \author[label]{name}.  This label should be identical to

% the corresponding label provided with the \address command

% N.B. It is not possible to link an author to more than one address.

%

\author[ICG]{Robert Viehauser, Christoph Bichler}

%----------------------------------------------------------------------

% List of addresses

%

% If there is more than one address, list each using a separate

% \address command using a label to link it to the respective author

% as described above


\newcommand{\TUGn}{Graz University of Technology}

\address[ICG]{Inst. for Computer Graphics and Vision \\ \TUGn, Austria}


%----------------------------------------------------------------------

% Information about the contact author

% if \contact is not defined (uncommented) or empty, the contact

%  information on the title page is suppressed.



% Name of contact

\contact{Robert Viehauser, Christoph Bichler}

% Email address of contact - do not use any LaTeX formatting here

\contactemail{robert.viehauser@student.tugraz.at, christoph.bichler@student.tugraz.at}

%----------------------------------------------------------------------

% Do not alter the following line



\begin{abstract}



This technical report is the result of the KU AKCV 2012/2013. Our task was to investigate
and evaluate the results and performance of a method presented in ~\cite{Singh2012DiscPat}.
This paper by Singh et al. presents a method for unsupervised discovery of \textit{discriminative patches}
which can serve as a mid-level visual representation. 
The discovered patches need to fullfil the following two requirements:
\begin{itemize}
 \item they need to be \textit{representative}, which means that they need to occur frequently enough in the visual world.
 \item they need to be \textit{discriminative}, which means that they need to differ enough from the rest of the visual world.
\end{itemize}

As stated before, these patches which represent the image should be found in an unsupervised manner. Therefore, Singh et al.
pose the problem of finding these patches as an unsupervised discriminative clustering problem on a huge dataset of image patches.



%----------------------------------------------------------------------

% Do not alter the following two lines

\end{abstract}

\section{Introduction}

In general, for object recognition it is desirable to find visual representations more compact than a full image to 
speed up the recognition task and minimize memory consumption. Examples for such ``sparse'' represenations are visual words,
bags of visual words or the deformable parts model. Singh et al. in \cite{Singh2012DiscPat} propose an \textit{unsupervised} method for finding mid-level
discriminative patches which according to them is not only more intuitive and reasonable for humans, but also offers very good discriminability,
broad coverage, better purity and improved performance to visual word features.\\
\\
Image patches used as features to represent an image may be chosen on a range of levels of representation.
To further quote from \cite{Singh2012DiscPat}, at the very low-level, bottom-up point of view an image patch simply represents
the appearance at its position in the image, either directly (with raw pixels \cite{Ulman2002VisualFeatures}) or transformed into a different representation (see \cite{Singh2012DiscPat} for examples).
A bit more high-level approach would be SIFT matching, which encodes patches at sparse interest points in a rotation- and scale-invariant 
way \cite{Lowe2004SIFT}. According to \cite{Singh2012DiscPat}, such bottom-up approaches work very well for instance recognition, but the results 
e.g. finding similar instance are not as good.\\
\\
As a result, researchers have started to go with more high-level features which unfortunately come with a number of significant practial barriers,
as listed in \cite{Sing2012DiscPat} like non-trivial amounts of hand-labeled training data per semantic entity (object, part etc.) and a lack
of visual discriminativeness to act as good features for some semantic entities.\\
\\
The algorithm proposed by Singh et al. extracts \textit{mid level} discriminative patches. These patches may correspond to parts, objects, ``visual phrases''
but don't necessarily have to. What is special about the method is that the patches are discovered in an unsupervised manner on just a pile of training images.
The patches are defined by their \textit{representative}(how often they occur in the visual world) and \textit{discriminative} property (how much they differ from the rest of the visual world).
To detect the patches, Singh et al. pose this as an unsupervised discriminative clustering problem on a huge unlabeled dataset of image patches.
They use an iterative procedure which alternates between clusering and training discriminative classifiers (using a linear SVM) while appyling 
cross-validation to prevent over-fitting.\\
\\
The authors of \cite{Singh2012DiscPat} also claim that their method can be used in a supervised setting and reaches state-of-the-art performance,
beating bag-of-words, spatial pyramdis \cite{Lazebnik2006SpatialPyr}, ObjectBank \cite{Li2010ObjectBank} and scene deformable-parts models \cite{Pandey2011PartBased} on the MIT Indoor-67 dataset \cite{MITIndoor}.\\
\\
In the paper \cite{doersch2012what}, Doersch et al. present an application of the described method. They extracted discriminative patches f
rom a huge database of images of Paris street scenes and then trained a classifier from them. As the authors say, these discriminative
patches are what makes Paris look like Paris. The classifier can be used to categorize street scenes to be Paris or non-Paris.
\section{Method / Implementation}

As stated before, Singh et al. posed the problem of finding discriminative patches as an unsupervised 
discriminative clustering problem on a huge unlabeled dataset of image patches. Therefore they use
an iterative procedure which alternates between clustering and training discriminative classifiers, while
applying cross-validation in each iteration step to prevent overfitting of the SVM cluster classifiers.\\
\\
Given an arbitrary set of unlabeled images (the discovery set $\mathcal{D}$), a number of \textit{discriminative patches}
at arbitrary resolution should be extracted which captures the ``essence'' of the data. For discrimination against
the real world, a set $\mathcal{N}$ of randomly sampled patches from a big number of images is used.\\
\\
In general, our implementation 
follows the one in \cite{Singh2012DiscPat}. We therefore assume that the reader is familiar with it.\\
\\
Therefore, we first extracted 1000 random patches (represented as by Singh et al. as HOG features with 9 orientations
and a cellsize of 8) from each image in the discovery set $\mathcal{D}$. We then pruned the set of patches such that its size
was at most the size of a parameter we introduced to handle the computation complexity. Then, we randomly permutated
the set of image patches and split it into sets $\mathcal{D_1}$ and $\mathcal{D_2}$ for training and validation. \\
\\
For an initial clustering we extracted a sample set of $\mathcal{S}$ patches from our set $\mathcal{D_1}$. We chose the size
of $\mathcal{S}$ to be $\frac{D_1}{10}$ and disallowed patches with very low or no gradient energy (like sky patches).
We then used k-means(the VLfeat implementation), with k = $\frac{S}{4}$ to obtain an initial clustering in HOG space and removed
clusters with less than 3 patches.\\
\\
Singh et al. state in \cite{Singh2012DiscPat}, that the only requirement the natural world set needs to fulfill is that
it is very large(thousands of images, containing tens of millions of patches) and drawn from a reasonably random image distribution.
Singh et al. therefore extracted a big number(6000) of random images from the Internet.\\
\\
For building our natural world set $\mathcal{N}$ we used patches extracted from the SUN2012 database \cite{SUN2010}.
This is a database consisting of 17.000 images from many different scene categories, showing many different objects with a lot of clutter.
We randomly picked 6000 out of the 17.000 images and randomly extracted a number of patches from each of the picked images to
ensure a random distribution of the image patches and computed their HOG descriptorrs. For our evaluation we originally
built a natural world set consisting of HOG descriptors of 300.000 patches, but soon had to discover that it was computationally unfeasible for us
to use a natural world set of this size with what little computational power we have. In \cite{Singh2012DiscPat} it is never clearly stated of how 
many patches their natural world set consists, but it is very probable that it is factors bigger, containing millions of patches.
Our world set $\mathcal{N}$ is also divided into to equally sized sets $\mathcal{N}_1$ and $\mathcal{N}_2$.\\
\\
In the iterative part of the algorithm we train a linear SVM classifier(using LibSVM-dense, C=0.1) for each cluster, using patches within
the cluster as positive examples and all patches in $\mathcal{N}_1$ as negative examples. The trained classifiers are then ran on the second part 
of the discovery set $\mathcal{D}_2$ and new clusters are formed from the top m firings of each detector (for our evaluation we used
m=5 and considered SVM scores above -1 as firings, as done in \cite{Singh2012DiscPat}). According to \cite{Singh2012DiscPat},
limiting the cluster size to 5 produces more heterogeneous clusters (patches from the same visual concept). Clusters which fire
only once or twice on the validation set are considered to be not very discriminative and are killed. The validation set and training
set are then swapped (so is $\mathcal{N}_1$ and $\mathcal{N}_2$) and the procedure is repeated until convergence (in practice after 4-5 iterations).

With a natural world set of 300.000 patches the runtime of training the classifier exploded in our implementation, so did the
memory usage.
Therefore we had to limit our natural world set and only tested our implementation and evaluated the results with a 
natural world set of 40.000 to 80.000 patches. Singh et al. used hard iterative mining to handle the complexity of $\mathcal{N}$,
but this did not result in a faster solution with our implementation.\\
\\
A summary of the procedure in pseudocode:\\
\\
\begin{verbatim}
   DISCOVER_TOP_N_DISCRIMINATIVE_PATCHES(nw_imgs_path, discovery_imgs_path)


   N = generate_world_set(nw_imgs_path);
   D = extract_patches(discovery_imgs_path);
   
   D=> {D1, D2};   N => {N1, N2};
   S <= random_sample(D1); 
   
   %Initial clustering using kmeans, k = S/4
   K <= kmeans(S);
   
   while not converged() do
      
      for all i such that size(K[i] >= 3) do
           %Iteratively train and cluster
           C_new[i] <= svm_train(K[i], N1);
           K_new[i] <= detect_top(C_new[i], D2, m);
      end for
      
      K <= K_new;
      C <= C_new;
      swap(D1,D2);
      swap(N1,N2);
      
   end while
   
   
   A[i] <= purity(K[i]) + lambda * discriminativeness(K[i]), for all i
   return select_top(C, A, n)
\end{verbatim}

\section{Evaluation}

As mentioned in the paper, evaluation of such an unsupervised method is quite difficult. As only qualitative measurement, they introduced terms of purity and coverage. During this project we also tried to implement these measurements over different discovery sets, but as stated before, we had problems with the runtime of our implementation on our available hardware even with one run.
Therefore, we could only test our implementation and the general method of discovering patches based on their visual
discriminativness with very basic tests.\\
\\
One test we conducted was a discovery set $\mathcal{D}$ of 15 images containing busses, which we trained with a worldset 
$\mathcal{N}$ of 80.000 patches. The whole calculation of the algorithm took about 10 hours due to the tremendous amount of patches, although the natural world set was precomputed. The results for the best clusters based on SVM score are visualized in the figure below (top 5 patches), 
the rows show the results per iteration.\\
\\
\begin{figure}[h!]
\centering
\includegraphics[width=0.9\textwidth]{figures/busses.png}
\caption{Per iteration results - Discriminative patches for busses}
\end{figure}
\\
As we can see, the top cluster is heterogeneous with patches from different visual concepts (parts of busses, but also a person).
To prevent this, Singh et al. encorporated a method to clean up heterogeneous cluster by finding what they called ``doublets'' \cite{Singh2012DiscPat}.\\
\\
Another test included a discovery set $\mathcal{D}$ of 10 images showing horses, trained with a natural worldset $\mathcal{N}$ of 40.000 patches.
The results for the best clusters based on the SVM score are shown below (top 5 patches), the rows again visualize the results per iteration.

\begin{figure}[h!]
\centering
\includegraphics[width=0.9\textwidth]{figures/horses.png}
\caption{Per iteration results - Discriminative patches for horses}
\end{figure}

The SVM scores serves as an approximation for the \textit{purity} of the cluster, i.e. the percentage of
how many members of the cluster come from the same visual concept. To calculate the \textit{discriminativness}
of a patch, we have to calculate the ratio of firings of the patch on $\mathcal{D}$ and firings of the patch on $\mathcal{D} \cup \mathcal{N}$.
Since the results in Figure 1 and 2 are solely based on the SVM score or purity, this measure is missing in this visualization.\\
Even the evaluation of purity and discriminativness of the learned clusters took about 5 hours.
\section{Conclusion}

As a conclusion we might say that while the method yields promising results in the paper it was introduced in 
(see \cite{Singh2012DiscPat}), it needs a lot of computational resources which might render the method unpractical.\\
\\
It was impossible for us to reproduce the results shown in the publication by Singh et al., but our evaluation results
might work as a proof of concept and could be improved by running the code on a cluster of computers with a much bigger
world set $\mathcal{N}$ and bigger discovery set $\mathcal{D}$.  

\bibliography{akcv}
\bibliographystyle{plain}

\end{document}

%----------------------------------------------------------------------

